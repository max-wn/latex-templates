% template - preamble for article
%\documentclass[a4paper,12pt]{article} % format of document, use it in 'main.tex' or here

\usepackage[T1,T2A]{fontenc}        % add eng,rus encoding support
\usepackage[utf8]{inputenc}         % add UTF8 support
\usepackage[russian,english]{babel} % add rus,eng(base) package
\usepackage{soul}                   % add l a t t r s p a c i n g
\usepackage{longtable}              % To display tables on several pages
\usepackage{booktabs}               % For prettier tables -> does not work
\usepackage{enumitem}               % advanced lists support
%\usepackage{fontspec}       % to use any font known to the operating system
%\setmainfont{PT Serif}      % set defolt font

\usepackage{amsmath, amsfonts, amssymb, amsthm, mathtools} % add math support

\usepackage{geometry}  % add document's fields correction support
\geometry{top=25mm}    % top field
\geometry{bottom=30mm} % bottom field
\geometry{left=20mm}   % left field
\geometry{right=20mm}  % right field

\linespread{1}               % length between str
\setlength{\parindent}{20pt} % red str
\setlength{\parskip}{12pt}   % length between paragraphs

\usepackage[backend=biber, style=authoryear-icomp]{biblatex} % add bibliography support
\addbibresource{$HOME/latex-templates/biblio.bib}            % path to bibliography base
\usepackage{csquotes}                                        % advanced facilities for inline and display quotations

\usepackage{indentfirst} % first paragraph with red str
\usepackage[colorinlistoftodos]{todonotes}   % allows us to write todonotes

% Must be the last command into the preamble of document.
\usepackage{hyperref} % All references in document turn into hyperlinks
\hypersetup{
unicode=true,      % Юникод в названиях разделов в PDF
colorlinks=true,   % Цветные ссылки вместо ссылок в рамках
linkcolor=blue,    % Внутренние ссылки
citecolor=green,   % Ссылки на библиографию
filecolor=magenta, % Ссылки на файлы
urlcolor=blue,     % Ссылки на URL
}
